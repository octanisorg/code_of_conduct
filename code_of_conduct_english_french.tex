%!TEX program = xelatex

\documentclass[12pt,a4paper,oneside]{article}
\usepackage{paracol}
\usepackage{fontspec}
\usepackage{microtype}
\usepackage{graphicx}
\graphicspath{ {images/} }
\usepackage[margin=2cm]{geometry}
\newcommand{\english}{    \switchcolumn[0]\noindent}
\newcommand{\french}{    \switchcolumn[1]\noindent}
\renewcommand{\thesubsubsection}{Art. \arabic{section}}


\setmainfont{Gibson Light}
\newfontfamily\semibold{Gibson}


\setcounter{section}{1}
\begin{document}

\begin{center}
	\includegraphics{octanis_org_logo_large}
\end{center}

\begin{paracol}{2}\sloppy


\english
	\section*{Code of Conduct}

\french
	\section*{Code de conduite}



\english
	\subsection{Introduction}
	{\semibold Octanis provides a place for curious and creative minds to embark on exploratory missions around the world. Our objective is to distill our findings into pragmatic solutions and inventions, accessible to everyone. }


	At Octanis, we tackle grand challenges which others only talk about. Members are confronted with long lasting and time consuming problems where any solution is unclear or even non-existent. This code provides guidance for Octanis members and board dealing with these issues.

	Every member of Octanis has an obligation to know and understand the guidelines contained within this code.
	Understanding and knowing about the code is not enough. You also have the obligation to comply with the values and spirit of this code and help other members do the same.

\french
	\subsection{Introduction}




\english
	\subsection{Responsibility}
	Have a sense of urgency and have the courage to question conventional wisdom. Be open minded and have the flexibility to broaden your experience by learning new skills. Be responsable for the progress of any tasks that you take on. If you are taking the lead of a project, you are responsible for looking after and guiding members who are helping you.



\french
	\subsection{Responsabilité}


\english
	\subsection{Excellence}
 	It is often necessary to spend a sustained period of time learning how to do something before actually achieving your goals. You are encouraged to learn and immediately apply your new knowledge in small steps. This will enable you to build something rapidly without knowing everything at the beginning of a project. You're encouraged to organise yourself and work independently. 
 	
 	We expect you to pay attention to details, but also regularly assess if the details are not distracting you from the goal.

	Members should never work in competition against each other. Neither should they step on each others areas of responsabilities or take over tasks without asking.




\french
	\subsection{...}

	
\english
	\subsection{Focus}
	At Octanis, we encourage members to take the lead of a single project or task and do that one thing well. Trying to work on fifteen different ideas at the same time will likely end in your own dissatisfaction. Focusing will help you achieve much more in a shorter amount of time. Ask members for help and set yourself a deadline (or ask a member to set one for you) and make a simple plan.



\french
	\subsection{...}

	

\english
	\subsection{Ideas}
	Good ideas are rare. Bad ideas are common. You'll have a hard time telling the difference if you keep everything to yourself.
	If you have an idea, write it down, then try convincing someone about it. We encourage members to share their ideas openly and get feedback quickly. Prevent groupthink by also getting feedback from your friends and family.

	If you don't have any ideas, talk to other members. Ask how you can help them with their projects. There are plenty of opportunities for new ideas to come up without you ever expecting it.

	If you incorporate ideas, tools or software from others, it's expected that you take their intellectual property rights into due consideration. Be sure to check licensing and always give credit appropriately.

	By default ideas created at Octanis are kept behind closed doors until they are deemed ready for public release. Members are not allowed to speak about ideas that are not theirs if they don't have permission to do so.

	Most importantly, don't talk or brag about your ideas or projects too much. Talking about doing something and never doing it is not what Octanis stands for. Don't over-analyse your idea. Be brave and try out your idea in any way you can and as fast as you can. «Minimise the time to try things. Maximise the rate of learning.»



\french
	\subsection{...}


\english
	\subsection{Failure}
	Failure is inevitable. You worked on a project for weeks and now realise it was all for nothing. You pitched your idea in front of an audience and got rejected. You were critised by a person you respect. 
	These can be painful events, but they all contain important messages: Do it again and do it fast.

	Sometimes the time isn't right, the people unprepared or the idea not thought through. It doesn't matter. What does is that you learn from your mistakes and get up and do it again. 



\french
	\subsection{...}


\english
	\subsection{Success}
	
	Success is not purely a matter of luck. You are not a lottery ticket. You will succeed if you put in hard work and encourage others to help you. 


	A successful Octanis project is one that has brought value to society. Value can take the form of helping others achieve their goals, reducing general suffering, improving learning and education, building great relationships, creating open devices, tools, protocols and methods, writing good documentation and so on.


\french
	\subsection{...}


\english
	\subsection{Respect}
 	We take pride in the diversity of our members, partners and others with whom we interact and treat them with respect, fairness and dignity. We are commited to create and maintain an environment free from discrimination, harassment and retaliation. 

 	When we criticise other views and ideas, we do so constructively and leave our own interests and ego out of the discussion. 
 	When we are criticised we listen and try to hear what the other person is saying. 
 	We try to refrain from unproductive and emotional arguments and we always come to an agreement.


\french
	\subsection{...}


\english
	\subsection{Moving on}
	If you don't want to work on a certain project anymore, make sure you tell the other members. Also inform them of why you are moving on and what you learned from the project work. Other members will respect your decision to move on in any case.

\french
	\subsection{...}



\end{paracol}
\end{document}
